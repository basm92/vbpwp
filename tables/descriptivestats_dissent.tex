\begin{table}[!h]

\caption{\label{tab:descriptivestats_dissent}Dissent in Voting Behavior in Key Laws}
\centering
\resizebox{\linewidth}{!}{
\fontsize{10}{12}\selectfont
\begin{tabular}[t]{lllrrrrrrrrr}
\toprule
\multicolumn{6}{c}{ } & \multicolumn{3}{c}{Party Line} & \multicolumn{3}{c}{Dissent} \\
\cmidrule(l{3pt}r{3pt}){7-9} \cmidrule(l{3pt}r{3pt}){10-12}
Category & Law & Year & N & Pct. In Favor & Status & Confessional & Liberal & Socialist & Confessional & Liberal & Socialist\\
\midrule
Electoral Law & Kieswet 1872 & 1874 & 71 & \num{0.45} & Rejected & Con & Pro & - & \num{0.04} & \num{0.30} & -\\
 & Kieswet 1887 & 1887 & 83 & \num{0.82} & Accepted & Pro & Pro & - & \num{0.39} & \num{0.02} & -\\
 & Kieswet 1892 & 1894 & 98 & \num{0.42} & Rejected & Con & Pro & Pro & \num{0.16} & \num{0.37} & \num{0.00}\\
 & Kieswet 1896 & 1896 & 88 & \num{0.74} & Accepted & Pro & Pro & Pro & \num{0.41} & \num{0.17} & \num{0.00}\\
 & Kieswet 1918 & 1919 & 68 & \num{0.85} & Accepted & Pro & Pro & Pro & \num{0.30} & \num{0.00} & \num{0.00}\\
Fiscal Law & Inkomstenbelasting 1872 & 1872 & 78 & \num{0.35} & Rejected & Con & Pro & - & \num{0.04} & \num{0.49} & -\\
 & Inkomstenbelasting 1893 & 1893 & 89 & \num{0.62} & Accepted & Con & Pro & Con & \num{0.26} & \num{0.08} & \num{0.00}\\
 & Inkomstenbelasting 1914 & 1914 & 80 & \num{0.85} & Accepted & Pro & Pro & Pro & \num{0.34} & \num{0.00} & \num{0.00}\\
 & Successiewet 1878 & 1878 & 80 & \num{0.60} & Accepted & Con & Pro & - & \num{0.04} & \num{0.10} & -\\
 & Successiewet 1911 & 1911 & 69 & \num{0.93} & Accepted & Pro & Pro & Pro & \num{0.14} & \num{0.00} & \num{0.00}\\
 & Successiewet 1916 & 1916 & 77 & \num{0.62} & Accepted & Con & Pro & Pro & \num{0.15} & \num{0.04} & \num{0.00}\\
 & Successiewet 1921 & 1921 & 70 & \num{0.77} & Accepted & Pro & Con & Pro & \num{0.26} & \num{0.17} & \num{0.00}\\
\bottomrule
\multicolumn{12}{l}{\rule{0pt}{1em}Dissent is defined as the percentage of politicians of each faction having voted against the party line.}\\
\multicolumn{12}{l}{\rule{0pt}{1em}Party Line is defined as the median vote per party: 'Pro' if in favor, 'Con' if against, 'None' if no discerible party line (equally split), and '-' if N.A.}\\
\multicolumn{12}{l}{\rule{0pt}{1em}Kieswet - Electoral Law, Inkomstenbelasting - Income Tax, Successiewet - Inheritance Tax}\\
\end{tabular}}
\end{table}
